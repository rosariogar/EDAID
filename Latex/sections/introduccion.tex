\documentclass[../main.tex]{subfiles}

\begin{document}

\subsection{Motivación}
La motivación de este proyecto está vinculada al interés de los miembros que componen el grupo por la Bioinformática. En este caso, se va a hacer un estudio de una lista de medicamentos encontrados en una gran farmaceútica conocida a nivel nacional llamada CIMA (Centro de Infromación de Medicamentos).

Para ello, se ha extraído un documento Excel de la página oficial, el cual lo componen datos suficientes para trabajar adecuadamente con él.


\subsection{Objetivos}
Los objetivos de este proyecto es reflejar de forma adecuada los conocimientos adquiridos en clase sobre XML, HTML, XSL, RDF, XQueries, Protegé, OWL, entre otras. Para poner en práctica todo esto, se ha seleccionado un archivo CSV con un conjunto de medicamentos y sus características, que son las siguientes: \textit{Número de Registro, Principios activos, Nombre del medicamento, Laboratorio titular, Prescripción, Vías de Administración, Forma Farmaceútica, Dosis y si se administran con Receta o no}. Este archivo se va a transformar a uno de tipo XML para poder realizar transformaciones XSLT y consultas XQueries, además de realizar un modelado en Protegé para poder visualizar su Grafo de Ontologías. 

Como objetivo final para este proyecto es la realización de una página Web en la cual podremos exhibir los medicamentos extraídos de dicho archivo con sus correspondientes características.


\subsection{Estructura del documento}
La estructura del documento va a constar de las siguientes partes: 
\begin{itemize}
    \item Una introducción que de una visión general de como se quiere enfocar el proyecto.
    \item Un apartado que se refiera a las consultas XQuery realizadas.
    \item Un apartado que se refiera a las transformaciones XSL del documento XML.
    \item Un apartado en el que se muestre el modelado de Protegé. 
    \item Una conclusión.
\end{itemize}

\subsection{Tecnologías usadas}
Las tecnologías usadas en este proyecto hasta el momento son las siguientes:
\begin{itemize}
    \item Aplicación de Uso libre para transformar el archivo CSV a XML.
    \item Visual Studio Code para las transformaciones XSLT.
    \item eXist-db para poder realizar las consultas XQueries.
    \item Protegé para poder diseñar un grafo que modele nuestros datos.
\end{itemize}

\end{document}